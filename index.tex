% Options for packages loaded elsewhere
\PassOptionsToPackage{unicode}{hyperref}
\PassOptionsToPackage{hyphens}{url}
\PassOptionsToPackage{dvipsnames,svgnames,x11names}{xcolor}
%
\documentclass[
  letterpaper,
  DIV=11,
  numbers=noendperiod]{scrreprt}

\usepackage{amsmath,amssymb}
\usepackage{iftex}
\ifPDFTeX
  \usepackage[T1]{fontenc}
  \usepackage[utf8]{inputenc}
  \usepackage{textcomp} % provide euro and other symbols
\else % if luatex or xetex
  \usepackage{unicode-math}
  \defaultfontfeatures{Scale=MatchLowercase}
  \defaultfontfeatures[\rmfamily]{Ligatures=TeX,Scale=1}
\fi
\usepackage{lmodern}
\ifPDFTeX\else  
    % xetex/luatex font selection
\fi
% Use upquote if available, for straight quotes in verbatim environments
\IfFileExists{upquote.sty}{\usepackage{upquote}}{}
\IfFileExists{microtype.sty}{% use microtype if available
  \usepackage[]{microtype}
  \UseMicrotypeSet[protrusion]{basicmath} % disable protrusion for tt fonts
}{}
\makeatletter
\@ifundefined{KOMAClassName}{% if non-KOMA class
  \IfFileExists{parskip.sty}{%
    \usepackage{parskip}
  }{% else
    \setlength{\parindent}{0pt}
    \setlength{\parskip}{6pt plus 2pt minus 1pt}}
}{% if KOMA class
  \KOMAoptions{parskip=half}}
\makeatother
\usepackage{xcolor}
\setlength{\emergencystretch}{3em} % prevent overfull lines
\setcounter{secnumdepth}{5}
% Make \paragraph and \subparagraph free-standing
\makeatletter
\ifx\paragraph\undefined\else
  \let\oldparagraph\paragraph
  \renewcommand{\paragraph}{
    \@ifstar
      \xxxParagraphStar
      \xxxParagraphNoStar
  }
  \newcommand{\xxxParagraphStar}[1]{\oldparagraph*{#1}\mbox{}}
  \newcommand{\xxxParagraphNoStar}[1]{\oldparagraph{#1}\mbox{}}
\fi
\ifx\subparagraph\undefined\else
  \let\oldsubparagraph\subparagraph
  \renewcommand{\subparagraph}{
    \@ifstar
      \xxxSubParagraphStar
      \xxxSubParagraphNoStar
  }
  \newcommand{\xxxSubParagraphStar}[1]{\oldsubparagraph*{#1}\mbox{}}
  \newcommand{\xxxSubParagraphNoStar}[1]{\oldsubparagraph{#1}\mbox{}}
\fi
\makeatother


\providecommand{\tightlist}{%
  \setlength{\itemsep}{0pt}\setlength{\parskip}{0pt}}\usepackage{longtable,booktabs,array}
\usepackage{calc} % for calculating minipage widths
% Correct order of tables after \paragraph or \subparagraph
\usepackage{etoolbox}
\makeatletter
\patchcmd\longtable{\par}{\if@noskipsec\mbox{}\fi\par}{}{}
\makeatother
% Allow footnotes in longtable head/foot
\IfFileExists{footnotehyper.sty}{\usepackage{footnotehyper}}{\usepackage{footnote}}
\makesavenoteenv{longtable}
\usepackage{graphicx}
\makeatletter
\newsavebox\pandoc@box
\newcommand*\pandocbounded[1]{% scales image to fit in text height/width
  \sbox\pandoc@box{#1}%
  \Gscale@div\@tempa{\textheight}{\dimexpr\ht\pandoc@box+\dp\pandoc@box\relax}%
  \Gscale@div\@tempb{\linewidth}{\wd\pandoc@box}%
  \ifdim\@tempb\p@<\@tempa\p@\let\@tempa\@tempb\fi% select the smaller of both
  \ifdim\@tempa\p@<\p@\scalebox{\@tempa}{\usebox\pandoc@box}%
  \else\usebox{\pandoc@box}%
  \fi%
}
% Set default figure placement to htbp
\def\fps@figure{htbp}
\makeatother
% definitions for citeproc citations
\NewDocumentCommand\citeproctext{}{}
\NewDocumentCommand\citeproc{mm}{%
  \begingroup\def\citeproctext{#2}\cite{#1}\endgroup}
\makeatletter
 % allow citations to break across lines
 \let\@cite@ofmt\@firstofone
 % avoid brackets around text for \cite:
 \def\@biblabel#1{}
 \def\@cite#1#2{{#1\if@tempswa , #2\fi}}
\makeatother
\newlength{\cslhangindent}
\setlength{\cslhangindent}{1.5em}
\newlength{\csllabelwidth}
\setlength{\csllabelwidth}{3em}
\newenvironment{CSLReferences}[2] % #1 hanging-indent, #2 entry-spacing
 {\begin{list}{}{%
  \setlength{\itemindent}{0pt}
  \setlength{\leftmargin}{0pt}
  \setlength{\parsep}{0pt}
  % turn on hanging indent if param 1 is 1
  \ifodd #1
   \setlength{\leftmargin}{\cslhangindent}
   \setlength{\itemindent}{-1\cslhangindent}
  \fi
  % set entry spacing
  \setlength{\itemsep}{#2\baselineskip}}}
 {\end{list}}
\usepackage{calc}
\newcommand{\CSLBlock}[1]{\hfill\break\parbox[t]{\linewidth}{\strut\ignorespaces#1\strut}}
\newcommand{\CSLLeftMargin}[1]{\parbox[t]{\csllabelwidth}{\strut#1\strut}}
\newcommand{\CSLRightInline}[1]{\parbox[t]{\linewidth - \csllabelwidth}{\strut#1\strut}}
\newcommand{\CSLIndent}[1]{\hspace{\cslhangindent}#1}

\KOMAoption{captions}{tableheading}
\makeatletter
\@ifpackageloaded{bookmark}{}{\usepackage{bookmark}}
\makeatother
\makeatletter
\@ifpackageloaded{caption}{}{\usepackage{caption}}
\AtBeginDocument{%
\ifdefined\contentsname
  \renewcommand*\contentsname{Table of contents}
\else
  \newcommand\contentsname{Table of contents}
\fi
\ifdefined\listfigurename
  \renewcommand*\listfigurename{List of Figures}
\else
  \newcommand\listfigurename{List of Figures}
\fi
\ifdefined\listtablename
  \renewcommand*\listtablename{List of Tables}
\else
  \newcommand\listtablename{List of Tables}
\fi
\ifdefined\figurename
  \renewcommand*\figurename{Figure}
\else
  \newcommand\figurename{Figure}
\fi
\ifdefined\tablename
  \renewcommand*\tablename{Table}
\else
  \newcommand\tablename{Table}
\fi
}
\@ifpackageloaded{float}{}{\usepackage{float}}
\floatstyle{ruled}
\@ifundefined{c@chapter}{\newfloat{codelisting}{h}{lop}}{\newfloat{codelisting}{h}{lop}[chapter]}
\floatname{codelisting}{Listing}
\newcommand*\listoflistings{\listof{codelisting}{List of Listings}}
\makeatother
\makeatletter
\makeatother
\makeatletter
\@ifpackageloaded{caption}{}{\usepackage{caption}}
\@ifpackageloaded{subcaption}{}{\usepackage{subcaption}}
\makeatother

\usepackage{bookmark}

\IfFileExists{xurl.sty}{\usepackage{xurl}}{} % add URL line breaks if available
\urlstyle{same} % disable monospaced font for URLs
\hypersetup{
  pdftitle={Methods for estimating Reproduction Number, R(t)},
  pdfauthor={Laura White, PhD},
  colorlinks=true,
  linkcolor={blue},
  filecolor={Maroon},
  citecolor={Blue},
  urlcolor={Blue},
  pdfcreator={LaTeX via pandoc}}


\title{Methods for estimating Reproduction Number, R(t)}
\author{Laura White, PhD}
\date{2025-02-14}

\begin{document}
\maketitle

\renewcommand*\contentsname{Table of contents}
{
\hypersetup{linkcolor=}
\setcounter{tocdepth}{2}
\tableofcontents
}

\bookmarksetup{startatroot}

\chapter*{Preface}\label{preface}
\addcontentsline{toc}{chapter}{Preface}

\markboth{Preface}{Preface}

Since the onset of the COVID-19 pandemic in early 2020, there has been a
proliferation of software packages that make inference about the current
state of an infectious disease outbreak based on daily counts of disease
cases. An important and widely used parameter is the instantaneous or
effective reproduction number, R(t), defined in Gostic et. al.~(2020) 1
as follows:

``The effective reproductive number, denoted as Re or Rt, is the
expected number of new infections caused by an infectious individual in
a population where some individuals may no longer be susceptible.''

Defined as such, R(t) is an unobserved quantity that captures the
aggregated combination of disease characteristics (e.g., infectiousness
under controlled conditions, mode of transport) and extrinsic factors
(e.g., lockdowns that reduce person-to-person contact). An R(t) = 1
indicates a ``stable'' epidemic at time t, where each infected person
infects on average one additional person; R(t) values above or below 1
represent a growing or a shrinking epidemic respectively. We focus on
R(t) given its relevance for time-sensitive decision-making, and
summarize the currently used inputs, data, methods and assumptions in
R(t) estimation across the following categories:

\begin{itemize}
\tightlist
\item
  How the relationship between R(t) and infections is defined
\item
  How R(t) is constrained using distributions for key variables
\item
  How R(t) is constrained over time
\item
  Additional data and distributions that are used to constrain R(t)
\item
  Inference frameworks that are used to estimate R(t) ** is constrained
  the right word?
\end{itemize}

In this paper we provide a theoretical comparison of the current field
of methods for estimating R(t), with the goal of informing user
decision-making about which package to choose and in interpreting
package outputs. For reference, Table 1 lists packages with an
accompanying peer-reviewed journal manuscript, Table 2 lists packages
without a peer-reviewed journal manuscript, and Table S1 contains R(t)
packages that calculate R(t) but were excluded from this summary. In the
text, the package citation is given the first time each package is
referenced.

We limit the methods discussed here to those for estimating historical
to present-day R(t) values using daily case count data, where a case can
be flexibly defined as an individual with a reported positive test
(either through healthcare-seeking behavior, routine surveillance, or a
hospital admission). Other methods not discussed here include inference
of R(t) exclusively from alternative data sources (e.g., genetic data,2
behavioral data,3 or viral loads in waste-water4), or calculations from
compartmental, agent-based models, or network.5--7 We also limit the
discussion to packages in the statistical software R,8 which may exclude
some packages in other software programs that combine many of the
methodological considerations discussed below.9 We do not discuss any
packages for now-casting or forecasting, though a number of R(t)
estimation packages can be used for this purpose. The methods discussed
below and references to specific R packages are current as of December
1, 2024. We attempt to harmonize the mathematical choices between each
package using terminology from each.

An evaluative comparison of the performance of these methods would be
highly complex, given the following challenges. Some of the most
widely-used packages are not accompanied with a peer-reviewed manuscript
that describes or evaluates the theory behind modeling choices. Each
package contains a subset of the methods below for constraining R(t) in
time, but with subtle variations in implementation that are often not
well-documented. Most packages have not been recently updated, and even
those that have are not maintained on CRAN, instead leaving updates on a
development version on GitHub. The combination of differing model
frameworks associated with each package make it challenging to easily
compare the impacts on estimated R(t), especially when considering
additional factors like ease of implementation and computational time

\bookmarksetup{startatroot}

\chapter{Introduction}\label{introduction}

There are two primary classes methods of estimating R(t) from case count
data that are used in most R software packages. The first class of
methods assumes there is a formulaic relationship between infections and
reproduction number, a relationship known as the renewal equation.10
These infections are then assumed to result in (some fraction of) the
observed cases. A second class of methods involves empirically
calculating a quantity that approximates the latent quantity represented
by a reproduction number by fitting a curve to the case count
time-series and finding the time-varying slope in log space (and then
performing other transformations). Empirical calculations are discussed
in detail below in our examination of ways in which R(t) is constrained
over time.

\section{Renewal equation estimates of
R(t)}\label{renewal-equation-estimates-of-rt}

The renewal equation relates R(t) and infections on day t, I(t), using a
third parameter known as the generation interval. The generation
interval, ω, is the time between infection in the infector and infection
in the infectee, and assuming independence is the linear combination of
incubation time, the time between infection and symptom onset in an
individual, and transmission time, the time between symptom onset in the
infector and infection of the infectee.11 A similar parameter to the
generation interval is the serial interval, which is the time between
symptom onset in the infector and symptom onset in the infectee. The
serial interval and generation interval are interchangeable if the
incubation time is independent from the transmission time, and some
formulations of the renewal equation use generation interval. In this
paper we use the generation interval ω described by a probability mass
function with non-zero values from day 1 (assuming that disease
incubation takes at least 1 day) to a maximum day s, i.e., the longest
interval between symptom onset in infector and infectee. Taking care to
note that R(t) is undefined on day 0 since there has been no
transmission yet (and assuming the initial infections are I(0)), the
formulation of the renewal equation is thus:

I(t)=R(t) ∑\_(i=max⁡(1, t-s+1))\^{}t▒ω(i) I(t-i) (Eq.1)

For brevity, we write the inner sum of (Eq.1) as:

\begin{verbatim}
Λ(t)=∑_(i=max⁡(1,   t-s+1))^t ω(i) I(t-i)           (Eq.2)  
\end{verbatim}

The assumptions of this formulation, as per Green et. al.~2022,12 are
that incident infections can be described deterministically within each
window of t∈{[}t-s+1,t{]} and that the generation interval distribution
does not change over the modeling time. A common reframing of the
renewal equation is to equate R(t) with an exponential growth rate, r.
Under specific conditions and within a small time window
(t∈{[}t-s+1,t{]}), infections can be assumed to grow exponentially at a
constant rate (r).12--14 Using Eq. 1 in the time window t∈{[}t- s+1,t{]}
and assuming some initial infections k, R(t) for t∈{[}t- s+1,t{]} can be
inferred from only r and ω: I(t)=ke\^{}rt,t∈{[}t-s+1,t{]} (Eq.3)
R(t)={[}∑\_(i=max⁡(1, t-s+1))\^{}t▒〖ω(i) e\^{}(-ri) 〗{]}\^{}(-1)
,t∈{[}t- s+1,t{]} (Eq.4) Again, we will omit the writing the bounds for
time in remaining formulae. A single R(t) value, say R\_0, can also be
put in the form of an infection attack rate, z,15 or in the final size
equation,16 to estimate the proportion of all individuals that were
affected by a disease with this R\_0:

z=1-exp⁡(-R\_0 z) (Eq.5)

The attack rate function and others are implemented in the package
epigrowthfit.17 The major difference between calculating R(t) from a
renewal equation or an exponential growth rate equation is whether I(t)
is used. If for a given time window both r and ω can be estimated
independently, then R(t) can be inferred without infection data.
Otherwise, infection data are needed to estimate R(t).

Using the renewal equation (Eq. 1) and given that I(t) and ω are known,
R(t) can be solved for algebraically starting with R(t=1) and iterating
forwards in time. However, this will produce highly volatile estimates
of R(t) that recover the incidence curve directly. This is undesirable
for several reasons: real-world infectivity likely does not vary
dramatically from day to day, and real-world infection data are rarely
complete, especially in an emerging epidemic, meaning that a certain
amount of uncertainty must be incorporated into any estimation
framework. In addition, infection incidence, I(t), are the data of
interest but it is impossible to observe, so many calculations instead
may use the observed reported cases, C(t), which requires some
additional processing to incorporate into calculations of R(t).
Therefore, a variety of constraints on R(t) are added in the inferential
process: using distributions on key variables, placing restrictions on
how R(t) varies through time, and with additional data sources and delay
distributions. These choices dictate which estimation framework is used,
which can add additional constraints.

\bookmarksetup{startatroot}

\chapter{Decision Tree}\label{decision-tree}

Here's where we will have the decision tree

\part{Packages}

Here's where we will talk about the packages

\chapter*{EpiNow2}\label{epinow2}
\addcontentsline{toc}{chapter}{EpiNow2}

\markboth{EpiNow2}{EpiNow2}

\chapter*{EpiEstim}\label{epiestim}
\addcontentsline{toc}{chapter}{EpiEstim}

\markboth{EpiEstim}{EpiEstim}

\chapter*{RtEstim}\label{rtestim}
\addcontentsline{toc}{chapter}{RtEstim}

\markboth{RtEstim}{RtEstim}

\part{Methods}

Here's where we will talk about the methods

\bookmarksetup{startatroot}

\chapter*{References}\label{references}
\addcontentsline{toc}{chapter}{References}

\markboth{References}{References}

\phantomsection\label{refs}
\begin{CSLReferences}{0}{1}
\end{CSLReferences}




\end{document}
